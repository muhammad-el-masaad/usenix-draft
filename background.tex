\section{Background: Circuit Locking}

\paragraph{Example} An example of a locking synthesis procedure is the following. Take the behavioral description $f$ and do conventional (locking-free) synthesis to obtain a gate-level implementation. Select $m$ internal wires in the resulting circuit uniformly at random, and associate each with a distinct key bit. Depending on whether the corresponding key bit is 0 or 1, insert either an XOR gate or an XOR gate along each wire, with one input of the gate connecting to the wire's driver and the other input dangling (this is illustrated in Figure ??). Replace each XNOR gate with an XOR gate followed by an inveror then use DeMorgan's laws to move these inverters to random locations up or down the netlist. Provide the resulting netlist as the synthesis output.

\newcommand{\dummyfig}[1]{
  \centering
  \fbox{
    \begin{minipage}[c][0.20\textheight][c]{0.45\textwidth}
      \centering{#1}
    \end{minipage}
  }
}

%% This part makes a figure
\begin{figure}[h]
  \dummyfig{Dummy Figure Label} 
  \caption{XOR locking.}
  \label{fig:dummy1}
\end{figure}

The reader can see that such an output satisfies the first correctness criterion in Defintion ??. With careful selection of wires in the second step, it is conceivable that it can be made to additionally satisfy the second correctness criterion. However, as we will see later, the procedure does not satisfy our condition for secure synthesis, and as such, must not be considered secure. We use this example locking procedure for its simplicity to illustrate how a basic locking procedure might work. The mechanism was proposed by \cite{} as part of a larger ``purely combinational" IC activation scheme and is known in the lierature as XOR locking \cite{}. It is worth noting that the authors specify that to minimize impact on the delay characteristics of the final chip, all critical paths in the netlist must be avoided when selecting wires for XOR insertion in Step ??.