\section{Old}
Story:
 ---- IP theft is a major issue due to outsourced fab
 
 ---- Logic locking/encryption/obfuscation, but we will cite Markov and stick with locking.
 
 ---- What is logic locking. (Figure, SG to send). Forward refs 
 to two threats that logic locking addresses: (1) IP theft; (2) over-building in the discussion section.
 
 ---- Several recent papers (NDSS'15, HOST'15) have shown that the master key can be recovered practically recovered ***If attacker can exercise design with inputs of his/her choosing and observe outputs.** But this attack model is unrealistic in several settings; for example, if the chip is being fabbed for the first time, or is not commercially available. 
 
 --- So far, the assumption has been that a weaker attacker who only has access to locked netlist can obtain no information about the master key. In this paper, we show that this is not the case. 
 
 
--- Logic synthesis is a procedure that transforms a behavioural description of Boolean functionality (written in Verilog, for instance) to a netlist of Boolean logic gates, i.e., a graph in which vertices are Boolean gates like NAND/NOR and edges are wires that connect gates. Commercial logic synthesis tools try to optimize for performance metrics like area, roughly the number of gates in the netlist. The vulnerability arises from the fact that all existing 
logic locking mechanisms in literature propose to insert key gates 
in a synthesized netlist (see Fig X of first logic locking paper) 
--- as we show in this paper, logic synthesis implicitly embeds information about the Boolean functionality in the netlist. 

--- Our contributions in this paper are two-fold: 
(1) we propose logic desynthesis, a 
powerful new attack (blurb). 
(2) In the context of this attack, we provide a 
precise 
characterization of ``secure synthesis" 
(3) concrete instantiation.







 
 
 
 
 
 
The traditional ``localized" IC supply chain is being replaced by a modern, globalized one. While once a single IC design company handled all steps in the chain from design to fabrication, different steps are now carried out by different entities. In particular, there is a trend to outsource the fabrication part of the process to off-shore manufacturing facilities, often referred to as ``fabs", and having the semiconductor company deal only with conceptual and physical design steps. These fabs could be physically located anywhere in the world, particularly in jurisdictions where intellectual property laws are lax  and/or weakly enforced.

These changes in the traditional flow have prompted research into potential IP infringement practices by foreign foundries that could threaten the semiconductor industry. The overbuilding, or building in excess, of chips by foundries for distribution in the black market is one such threat: IC foundries are typically licensed to fabricate only a limited number of chips from the photo-masks provided to them by designers. Any production of ICs beyond the licenced amount is considered an IP infringement, and could be punishable by law if detected. The problem is, once a designer hands over a plain photo-mask to the fab, they no more have control over how many chips the fab decides to produce. This, of course, does not work in favor of IC design companies, whose IP is being exploited without their consent. 

Logic encryption is one technique that have recently been proposed to protect designer IPs from overproduction by foundries. The technique works by modifying the IC design flow in a way that requires a foundry to communicate with the IC design company every time they wish to release a newly fabricated and functioning chip into the market. The basic idea is to modify the gate-level netlist of the design so it 
%that the photomask received by the fab produces a chip which 
can perform one of many functions; depending on the value programmed into a \emph{key register}. When the correct key is stored into the key register, the chip performs performs its original function; otherwise it performs a potentially different function, depending on how the original gate-level netlist modified. The correct key value is held secret by the IP rights holder and released only in an encrypted form to the fab, at the time when the fab wants to distribute a newly manufactured chip on the market. When the fab gets the encrypted key from the IP holder, it enters it into the chip, where the key gets decrypted to its plain form by specially added cryptographic modules\footnote{The decryption key is unique for each chip, to prevent the fab from using the same key to activate multiple chips}; bringing the chip to a working state.
%; in a process referred to as \emph{activation}. 
Without knowledge of the correct key for a chip, and with no prior information on the design, the fab can only release the chip with a random value in the key register, which may cause the chip to behave differently than intended. By necessitating communication between fab and designer before a new, correctly functioning chip can be released on the market, the designer is enabled to keep track of the amount of chips produced by the fab; which limits the fab's ability to overbuild. 

%The ``correct" encrypted form is unique for each chip, as To prevent a fab from using the same key to activate multiple chips, the register-transfer level (RTL) description is also ``enriched" with support for an on-chip true random number generator (TRNG) that produces a randomized chip IP upon power-up. This ID must be sent by the fab to the IP rights holder which uses it to encrypt the locking key. requires its own encrypted key to be applied before it can function., in a process known as ``activation". Once the correct key appears on the key register, the chip can now be released on the market, as it can now perform its intended function. The same key cannot be used When the designer provides the fab with this special ``unlocking" code, it authorizes the release of the new chip. By doing this, the designer now has a means to keep track of the number of chips produced by the foundry, and the fab's ability to build in excess is effectively hindered. To prevent the fab from using the same code to activate more than one chip, the non-functioning version of the chip is designed such he activation codes are unique for each chip, . This is done by utilizing process variations in the form of a physically uncloneable function. 


%Central to the logic encryption piracy prevention technique 
Arguably the most important step in the new, logic-encrypted flow is what is known as \emph{logic locking}, the step in the flow where the gate-level netlist of the design is modified to obtain the new, ``locked" netlist. Almost all logic encryption technqius that have been proposed in the literature work on an intermediate representation of the design (in the form of a BLIF file), but they do not ensure a critical condition; that the encrypted netlist itself does reveal any information about the correct key. given that most of them do not provide any computational guarantees of hardness of key recovery from encrypted design and knowledge of encryption procedure, in this paper, We aruge that such information-theoretic guarantee is necessary. (unlike the case in conventional cryptography, this is harder to do for VLSI designs than one might initially think %/This is however not trivial to accomplish 
when limits are placed on the key length). To this end, we propose a new encryption procedure that enables easy locking of the design and at the same time ensures absolute independence of the ``encrypted" design and the correct key. We formalize our notion of security in terms of two conditions and argue that any ``encryption" procedure must satisfy these conditions at a minimum for it to be considered secure (provided that the procedure does not provide computational hardness guarantees). To motivate our technique and notion of security, we also report experimental results that illustrate the fact that most combinational locking considered ``secure" do not in fact have the minimal properties that would be expected from a secure scheme. To give an insight on the overhead resulting from our locking scheme, we also report area and delay characteristics of benchmarks when locked using our technique.

\subsection{Are Existing Schemes Secure?}
In Section ??, we report experimental evidence that a sample locking procedure --- XOR locking --- is not secure as a simple attack manages to recover more bits of the correct key that should be possible by random chance. Here, we make make an informal argument why --- due to the way they work in general --- none of the existing locking schemes satisfies our necessary condition for secure synthesis.

A quick look at locking schemes that have been proposed in the literature reveals a general trend to refrain from perform locking on the RTL; but to rather do it on some a technology-mapped netlist (as in the case ofo \cite{} and \cite{}), or unspecified (but implied to be a network, as in \cite{}). This to justifiable to some extent: one can argue that locking is more easily done on a network-representation of a circuit; rather than an abstract form of desired behaviour (such as a Verilog or VHDL code). To see this, consider that the opportunities for modification are generally higher in a Boolean network than in a corresponding RTL description (a single Verilog line of code such as z = a + b where a and b are 32-bit registers will be converted by a Verilog compiler to a network with at least 32 XOR gates, each of which is candidate for modification by a locking algorithm, e.g. one that inserts XOR gate). The second correctness criterion is also easier to satisfy when working on Boolean network rather than RTL. For example in the context of XOR locking, one can use an automatic test pattern generation tool to make sure that two stuck-at-faults at the selected wires do not cancel each other out, thereby rendering some incorrect key equivalent to the correct one, in the sense that applying the incorrect key results in correct circuit behaviour. Given that ATPG tools accept netlists as input, it is attractive to defer locking till at least some form of synthesis has been performed (e.g, until a corresponding multi-level network has been constructed by the synthesis tool, as a first step to prepare the network for optimization).

In this paper, we argue that performing locking post-synthesis (or in general, after some optimizing transformation has been applied on the RTL) in a logic locking scheme, introduces a subtle vulnerability in the scheme. The vulnerability comes about as a result of the fact that --- Unlike ``raw" RTL (pre-synthesis) --- in an optimized network, different parts of the network are no longer ``independent" of each other as they were before synthesis. Consider, for instance, an optimizing transformation that looks for sequences of inverters in a netlist, and replaces each with a single (or no) inverter (preserving the Boolean function implemented by the network). This is a very simple transformation to apply on a circuit netlist, and one that presumably any worthwhile logic synthesis tool should implicitly or explicitly perform. Now assume that the step of moving inverters up and down the netlist in the XOR locking scheme is removed from the procedure. A sequence of two inverters will then be a dead giveaway that the correct value of the key bit feeding the preceding XOR gate is a 1 (and so an inverter has been placed by the locking procedure after the XOR gate to ensure the first correctness criterion is satisfied). This is of course a trivial example, and it only fair to assume that the authors of the XOR locking procedure introduced the inverter moving step to prevent such an easy attack. Our point is that unless the procedure guarantees \emph{by construction}, and \emph{for every circuit}, that each key is equally likely regardless of the procedure's output (as ensured by our necessary condition for secure synthesis), such a locking procedure should \emph{not} be considered secure.

\begin{table*}
\caption{Results for the hill-climbing MSE attack on c880. }
\begin{tabular}{ | c | c | c | }
\hline
Key length & Number of correctly recovered bits & Time taken (s) \\
\hline
32-bit & 16 & 431.3 \\
& 14 & 389.4\\
& 15 & 401.6\\
& 20 & 380.7\\
& 17 & 380.6\\
Avg. & 16.4 &\\
\hline
64-bit & 31 & 929.4\\
& 37 & 1361.5\\
& 36 & 1442.1\\
& 33 & 1346.6\\
& 34 & 1299.1\\
Avg. & 34.2 & \\
\hline
\end{tabular}
\end{table*}

\begin{table*}
\caption{Output corruptibility numbers for the hill-climbing MSE attack on c880 obfuscated with 32 key gates. }
\begin{tabular}{ | c | c | p{5cm} | c | }
\hline
Average Corruptibility  & Corruptibility of recovered key  & No. of correctly recovered bits & MSE\\
\hline
9.68 & 11.73 & 15 & 81\\

11.17 & 9.54 & 21 & 14\\

12.18 & 7.58 & 18 & 74\\

13.21 & 14.58 & 16 & 32\\

11.60 & 11.73 & 18 & 50\\
\hline
\end{tabular}
\end{table*}

\begin{table}
\caption{Results for the hill-climbing attack based on the absolute difference in number of nodes and edges on c880}
\begin{tabular}{ | c | c | c | }
\hline
Key length & Number of correctly recovered bits & Time taken (s) \\
\hline
32-bit & 23&                                                         1645.87\\
&24&                                                         1458.24\\
&27&                                                         1667.95\\
&25&                                                         1653.95\\
&25&                                                         1813.47\\
Avg. & 24.8 &\\
\hline
64-bit & 48 &                                                         5385.41 \\
& 44 &                                                         4790.02 \\
&48&                                                         4686.72 \\
&48&                                                         5035.48 \\
& 47&                                                         3780.07\\
Avg. & 47&\\
\hline
\end{tabular}
\end{table}

\begin{table*}
\caption{Results for the absolute-delay-difference based exhaustive and hill-climbing attacks on c880 obfuscated with 12 and 32 key gates. Delay difference is between observed netlist and the one resulting from unmapping the network then re-mapping it. }
\begin{tabular}{ | c | c | p{5cm} | c | }
\hline
Key length Average Corruptibility  & Corruptibility of recovered key  & No. of correctly recovered bits \\
\hline
12-bit & 5.48 & 2.12 & 10\\

& 5.44 & 1.19 & 10\\

& 5.33 & 3.19 & 8\\

& 5.19 & 6 & 7\\

& 4.81 & 1.19 & 10\\
\hline
32-bit & 5.48 & 2.12 & 10\\

& 11.48 & 9.54 & 22\\

& 12.25 & 11.46 & 19\\

& 10.11 & 4.38 & 23\\

& 12.16 & 11 & 18\\
\hline
\end{tabular}
\end{table*}

\begin{table}
\caption{Results for the delay-based exhaustive attack on c880 (The \emph{choice; map; ps} command sequence of the ABC synthesis tool is repeatedly executed until we get three consecutive delay increases or the area stops improving)}
\begin{tabular}{ | c | c | c | c | }
\hline
Hamming distance & 0 & 2 & 5 \\
&1& 4 & 3 \\
&1& 3 & 2 \\
&2& 1 & 2 \\
&3& 1 & 4 \\
\hline
Key Size & 10-bit & 11-bit & 12-bit \\
\hline
\end{tabular}
\end{table}

\begin{table*}
\caption{Performance of hill-climbing MSE attack on 11 sequential benchmarks from the MCNC set. Numbers are average number of key bits correctly recovered by attack for 5 runs on each benchmark. }
\begin{tabular}{ | c | c | c | c | c | c | c | c | c | }
\hline
 &	\multicolumn{8}{ | c | }{Key Size} \\
 \hline
Circuit Name &	32-bit (MSE) &  & 32-bit (n+d) & 64-bit (MSE) & & 64-bit (n+d) &	128-bit (MSE) & 128-bit (n+d) \\
\hline
dk16	&   20.5    &  24.5 &  19.8 &  &   &   &   & 		 \\
ex1	    &   17      &  23.7 & 17.6 & & & & & 		\\
keyb &	17.5 & 22.4 & 16.4 & & & & & 		\\
planet1 &	20 & 21.13 & 15.8 &	39 & 44 & 31.6 & & 	\\
planet &	19.6 & 20.8 & 14.8 &	33	& & 31.4 & &   \\
s1a	& 20	& & 16.4 & & & &	& \\
s1	& 18 & & 16.8 &	37 & & 31.2 & & \\
sand	& 17.8 & & 13 &	36.6 & & 34.2 & & \\	
scf	& 21.2	& & 15.4 & 40.2 & & 33.4	& 75 & 63.8 \\
styr &	18.8 & &	16.2 & 38.6 & & 30 & &  \\	
tbk	& 22.2 & & 17.6 &	36 & & 34.2 & & \\	
\hline
\multicolumn{9}{ | c | }{In all cases for the experiment where I run the attack 100 times, the number of recovered key bits is greater than 50.} \\
\hline
\end{tabular}
\end{table*}

\begin{table}[ht]
\centering
\caption{Results from preliminary implementation of desynthesis attack. Locations of key gates are 
picked at random in each experiment, and the experiment is repeated five times. We report the minimum, maximum and average number of key bit recovered correctly, and the average time (in seconds) that the attack took.}
\label{my-label}
\begin{tabular}{|l|l|l|l|l|l|l|l|l|l|l|l|l|}
\hline
\multirow{4}{*}{} & \multicolumn{12}{c|}{Number of Key Bits}                                                                                                                        \\ \cline{2-13} 
                  & \multicolumn{4}{c|}{32}                             & \multicolumn{4}{c|}{64}                             & \multicolumn{4}{c|}{128}                            \\ \cline{2-13} 
                  & \multicolumn{3}{l|}{Correctly Recovered} &          & \multicolumn{3}{l|}{Correctly Recovered} &          & \multicolumn{3}{l|}{Correctly Recovered} &          \\ \cline{2-13} 
                  & Min         & Max         & Avg.         & Time (s) & Min         & Max         & Avg          & Time (s) & Min         & Max         & Avg          & Time (s) \\ \hline
alu4              & 17          & 20          & 18.2         &     1179.66     & 39          & 46          & 42.4         &      3562.4    & 77          & 90          & 84.2         &    10823.74      \\ \hline
c880              & 20          & 23          & 21.2         &      1647.896    & 43          & 45          & 44           &    4735.54      & 82          & 94          & 91.2         &    7947.628      \\ \hline
s1494             & 18          & 19          & 18.4         &      667.4    & 48          & 50          & 47.8         &    2575.932      & 86          & 89          & 86.4         & 7711.1         \\ \hline
c1908             & 14          & 22          & 18.2         &      520.66    & 36          & 39          & 47.8         &    1365.758     & 86          & 93          & 89.8         & 5553.688         \\ \hline
c3540             & 18          & 24          & 18.2         &      1005.52    & 30          & 41          & 37.4         &    2732.614     & 68          & 82          & 75.2         & 9499.594         \\ \hline
\end{tabular}
\end{table}

\begin{table*}
\caption{Area and delay overheads of our BDD-based obfuscation technique when implemented on the alu4 benchmark. }
\begin{tabular}{ | c | c | c | c | c | }
\hline
 Key size (bits) & Unobfuscated & 5 & 16 & 32 \\
\hline
Number of gates & 620 & - & 1048 & 1360 \\
Area & 1.0X & $\approx$ 1.0X & 1.7X & 2.19X \\
Delay & 1.0X & 1.13 & 1.6X & 1.3X \\
\hline
\end{tabular}
\end{table*}

\begin{table*}
\caption{Area and delay overheads of our BDD-based obfuscation technique when implemented on the alu4 benchmark. }
\begin{tabular}{ | c | c | c | c | c | }
\hline
 Benchmark Name & Area Overhead (32-bit) & Delay Overhead (32-bit)  & Area Overhead (64-bit) & Delay Overhead (64-bit) \\
\hline
bbara.kiss2&3.91&1.36&5.72&1.68\\
bbsse.kiss2&2.22&1.27&3.16&1.42\\
beecount.kiss2&4.04&1.88&11.60&2.36\\
cse.kiss2&1.72&1.09&0.00&0.00\\
dk14.kiss2&2.55&1.50&4.08&1.60\\
dk15.kiss2&3.56&1.32&5.88&1.70\\
dk16.kiss2&1.46&1.16&1.94&1.24\\
dk17.kiss2&3.49&1.60&5.69&1.69\\
dk512.kiss2&3.37&1.45&5.35&1.58\\
donfile.kiss2&1.75&1.36&2.66&1.38\\
ex1.kiss2&1.68&1.13&2.28&1.20\\
ex2.kiss2&2.13&1.38&3.08&1.53\\
ex3.kiss2&3.23&1.53&5.32&1.58\\
ex4.kiss2&2.76&1.48&4.31&1.69\\
ex5.kiss2&3.11&1.41&5.18&1.63\\
ex6.kiss2&2.96&1.30&4.29&1.42\\
ex7.kiss2&3.34&1.47&5.40&1.60\\
keyb.kiss2&1.70&1.08&2.27&1.15\\
kirkman.kiss2&2.17&1.17&3.02&1.23\\
mark1.kiss2&2.66&1.55&3.68&1.84\\
opus.kiss2&2.61&1.41&4.06&1.52\\
planet1.kiss2&1.26&1.04&1.61&1.22\\
planet.kiss2&1.27&1.27&1.61&1.26\\
s1a.kiss2&1.45&1.15&1.87&1.22\\
s1.kiss2&1.33&1.10&1.66&1.18\\
sand.kiss2&1.31&1.13&1.56&1.22\\
scf.kiss2&1.14&1.06&1.33&1.04\\
sse.kiss2&2.29&1.24&3.20&1.47\\
styr.kiss2&1.36&1.03&1.72&1.09\\
tbk.kiss2&1.34&1.00&1.47&1.10\\
train11.kiss2&3.95&1.72&&\\
\hline
\end{tabular}
\end{table*}

\begin{table}
\caption{Results for the delay-based exhaustive attack on c880 (The \emph{choice; map; ps} command sequence of the ABC synthesis tool is repeatedly executed until we get three consecutive delay increases or the area stops improving)}
\begin{tabular}{ | c | c | c | c | }
\hline
Hamming distance & &  & 4 \\
&&  & 3 \\
&&  & 3 \\
&&  & 3 \\
&&  & 2 \\
\hline
Key Size & 10-bit & 11-bit & 12-bit \\
\hline
\end{tabular}
\end{table}

\begin{table*}[ht]
\centering
\caption{Performance of hill-climbing MSE attack on 11 sequential benchmarks from the MCNC set. Numbers are average number of key bits correctly recovered by attack for 5 runs on each benchmark.}
\label{my-label}
\begin{tabular}{|l|l|l|l|l|l|l|l|l|l|l|l|l|}
\hline
\multirow{4}{*}{} & \multicolumn{12}{c|}{Number of Key Bits}                                                                                                                        \\ \cline{2-13} 
                  & \multicolumn{4}{c|}{32}                             & \multicolumn{4}{c|}{64}                             & \multicolumn{4}{c|}{128}                            \\ \cline{2-13} 
                  & \multicolumn{3}{l|}{Correctly Recovered} &          & \multicolumn{3}{l|}{Correctly Recovered} &          & \multicolumn{3}{l|}{Correctly Recovered} &          \\ \cline{2-13} 
                  & Min         & Max         & Avg.         & Time (s) & Min         & Max         & Avg          & Time (s) & Min         & Max         & Avg          & Time (s) \\ \hline
dk16              &           &           & 24.5         &          &           &           &         44 &          &           &           &          &          \\ \hline
ex1              &           &           & 23.7         &          &           &           &            &          &           &           &          &          \\ \hline
keyb             &           &           & 22.4         &          &           &           &          &          & 
&           &          &          \\ \hline
planet1             &           &           & 21.13         &          &           &           &          &         &           &           &          &          \\ \hline
planet             &           &           & 20.8         &          &           &           &      33    &         &           &           &          &          \\ \hline
s1a             &           &           & 20         &          &           &           &          &      &           &           &          &          \\ \hline
s1             &           &           & 18          &          &           &           &         37 &         &           &           &          &          \\ \hline
sand             &           &           & 17.8         &          &           &           &        36.6  &         &           &           &          &          \\ \hline
scf             &           &           & 21.2         &          &           &           &         40.2 &         &           &           &         75 &          \\ \hline
styr             &           &           & 18.8         &          &           &           &         38.6 &         &           &           &          &          \\ \hline
tbk             &           &           & 22.2         &          &           &           &         36 &         &          &           &          &          \\ \hline

\end{tabular}
\end{table*}
