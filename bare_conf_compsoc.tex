
%% bare_conf_compsoc.tex
%% V1.4b
%% 2015/08/26
%% by Michael Shell
%% See:
%% http://www.michaelshell.org/
%% for current contact information.
%%
%% This is a skeleton file demonstrating the use of IEEEtran.cls
%% (requires IEEEtran.cls version 1.8b or later) with an IEEE Computer
%% Society conference paper.
%%
%% Support sites:
%% http://www.michaelshell.org/tex/ieeetran/
%% http://www.ctan.org/pkg/ieeetran
%% and
%% http://www.ieee.org/

%%*************************************************************************
%% Legal Notice:
%% This code is offered as-is without any warranty either expressed or
%% implied; without even the implied warranty of MERCHANTABILITY or
%% FITNESS FOR A PARTICULAR PURPOSE! 
%% User assumes all risk.
%% In no event shall the IEEE or any contributor to this code be liable for
%% any damages or losses, including, but not limited to, incidental,
%% consequential, or any other damages, resulting from the use or misuse
%% of any information contained here.
%%
%% All comments are the opinions of their respective authors and are not
%% necessarily endorsed by the IEEE.
%%
%% This work is distributed under the LaTeX Project Public License (LPPL)
%% ( http://www.latex-project.org/ ) version 1.3, and may be freely used,
%% distributed and modified. A copy of the LPPL, version 1.3, is included
%% in the base LaTeX documentation of all distributions of LaTeX released
%% 2003/12/01 or later.
%% Retain all contribution notices and credits.
%% ** Modified files should be clearly indicated as such, including  **
%% ** renaming them and changing author support contact information. **
%%*************************************************************************


% *** Authors should verify (and, if needed, correct) their LaTeX system  ***
% *** with the testflow diagnostic prior to trusting their LaTeX platform ***
% *** with production work. The IEEE's font choices and paper sizes can   ***
% *** trigger bugs that do not appear when using other class files.       ***                          ***
% The testflow support page is at:
% http://www.michaelshell.org/tex/testflow/



\documentclass[conference,compsoc]{IEEEtran}
% Some/most Computer Society conferences require the compsoc mode option,
% but others may want the standard conference format.
%
% If IEEEtran.cls has not been installed into the LaTeX system files,
% manually specify the path to it like:
% \documentclass[conference,compsoc]{../sty/IEEEtran}





% Some very useful LaTeX packages include:
% (uncomment the ones you want to load)


% *** MISC UTILITY PACKAGES ***
%
%\usepackage{ifpdf}
% Heiko Oberdiek's ifpdf.sty is very useful if you need conditional
% compilation based on whether the output is pdf or dvi.
% usage:
% \ifpdf
%   % pdf code
% \else
%   % dvi code
% \fi
% The latest version of ifpdf.sty can be obtained from:
% http://www.ctan.org/pkg/ifpdf
% Also, note that IEEEtran.cls V1.7 and later provides a builtin
% \ifCLASSINFOpdf conditional that works the same way.
% When switching from latex to pdflatex and vice-versa, the compiler may
% have to be run twice to clear warning/error messages.






% *** CITATION PACKAGES ***
%
\ifCLASSOPTIONcompsoc
  % IEEE Computer Society needs nocompress option
  % requires cite.sty v4.0 or later (November 2003)
  \usepackage[nocompress]{cite}
\else
  % normal IEEE
  \usepackage{cite}
\fi
% cite.sty was written by Donald Arseneau
% V1.6 and later of IEEEtran pre-defines the format of the cite.sty package
% \cite{} output to follow that of the IEEE. Loading the cite package will
% result in citation numbers being automatically sorted and properly
% "compressed/ranged". e.g., [1], [9], [2], [7], [5], [6] without using
% cite.sty will become [1], [2], [5]--[7], [9] using cite.sty. cite.sty's
% \cite will automatically add leading space, if needed. Use cite.sty's
% noadjust option (cite.sty V3.8 and later) if you want to turn this off
% such as if a citation ever needs to be enclosed in parenthesis.
% cite.sty is already installed on most LaTeX systems. Be sure and use
% version 5.0 (2009-03-20) and later if using hyperref.sty.
% The latest version can be obtained at:
% http://www.ctan.org/pkg/cite
% The documentation is contained in the cite.sty file itself.
%
% Note that some packages require special options to format as the Computer
% Society requires. In particular, Computer Society  papers do not use
% compressed citation ranges as is done in typical IEEE papers
% (e.g., [1]-[4]). Instead, they list every citation separately in order
% (e.g., [1], [2], [3], [4]). To get the latter we need to load the cite
% package with the nocompress option which is supported by cite.sty v4.0
% and later.






% *** GRAPHICS RELATED PACKAGES ***
%
\ifCLASSINFOpdf
  \usepackage[pdftex]{graphicx}
  % declare the path(s) where your graphic files are
  % \graphicspath{{../pdf/}{../jpeg/}}
  % and their extensions so you won't have to specify these with
  % every instance of \includegraphics
  % \DeclareGraphicsExtensions{.pdf,.jpeg,.png}
\else
  % or other class option (dvipsone, dvipdf, if not using dvips). graphicx
  % will default to the driver specified in the system graphics.cfg if no
  % driver is specified.
  % \usepackage[dvips]{graphicx}
  % declare the path(s) where your graphic files are
  % \graphicspath{{../eps/}}
  % and their extensions so you won't have to specify these with
  % every instance of \includegraphics
  % \DeclareGraphicsExtensions{.eps}
\fi
% graphicx was written by David Carlisle and Sebastian Rahtz. It is
% required if you want graphics, photos, etc. graphicx.sty is already
% installed on most LaTeX systems. The latest version and documentation
% can be obtained at: 
% http://www.ctan.org/pkg/graphicx
% Another good source of documentation is "Using Imported Graphics in
% LaTeX2e" by Keith Reckdahl which can be found at:
% http://www.ctan.org/pkg/epslatex
%
% latex, and pdflatex in dvi mode, support graphics in encapsulated
% postscript (.eps) format. pdflatex in pdf mode supports graphics
% in .pdf, .jpeg, .png and .mps (metapost) formats. Users should ensure
% that all non-photo figures use a vector format (.eps, .pdf, .mps) and
% not a bitmapped formats (.jpeg, .png). The IEEE frowns on bitmapped formats
% which can result in "jaggedy"/blurry rendering of lines and letters as
% well as large increases in file sizes.
%
% You can find documentation about the pdfTeX application at:
% http://www.tug.org/applications/pdftex





% *** MATH PACKAGES ***
%
\usepackage{amsmath}
\interdisplaylinepenalty=2500
\usepackage{mathrsfs}
% A popular package from the American Mathematical Society that provides
% many useful and powerful commands for dealing with mathematics.
%
% Note that the amsmath package sets \interdisplaylinepenalty to 10000
% thus preventing page breaks from occurring within multiline equations. Use:
%\interdisplaylinepenalty=2500
% after loading amsmath to restore such page breaks as IEEEtran.cls normally
% does. amsmath.sty is already installed on most LaTeX systems. The latest
% version and documentation can be obtained at:
% http://www.ctan.org/pkg/amsmath





% *** SPECIALIZED LIST PACKAGES ***
%
%\usepackage{algorithmic}
% algorithmic.sty was written by Peter Williams and Rogerio Brito.
% This package provides an algorithmic environment fo describing algorithms.
% You can use the algorithmic environment in-text or within a figure
% environment to provide for a floating algorithm. Do NOT use the algorithm
% floating environment provided by algorithm.sty (by the same authors) or
% algorithm2e.sty (by Christophe Fiorio) as the IEEE does not use dedicated
% algorithm float types and packages that provide these will not provide
% correct IEEE style captions. The latest version and documentation of
% algorithmic.sty can be obtained at:
% http://www.ctan.org/pkg/algorithms
% Also of interest may be the (relatively newer and more customizable)
% algorithmicx.sty package by Szasz Janos:
% http://www.ctan.org/pkg/algorithmicx




% *** ALIGNMENT PACKAGES ***
%
%\usepackage{array}
% Frank Mittelbach's and David Carlisle's array.sty patches and improves
% the standard LaTeX2e array and tabular environments to provide better
% appearance and additional user controls. As the default LaTeX2e table
% generation code is lacking to the point of almost being broken with
% respect to the quality of the end results, all users are strongly
% advised to use an enhanced (at the very least that provided by array.sty)
% set of table tools. array.sty is already installed on most systems. The
% latest version and documentation can be obtained at:
% http://www.ctan.org/pkg/array


% IEEEtran contains the IEEEeqnarray family of commands that can be used to
% generate multiline equations as well as matrices, tables, etc., of high
% quality.




% *** SUBFIGURE PACKAGES ***
%\ifCLASSOPTIONcompsoc
%  \usepackage[caption=false,font=footnotesize,labelfont=sf,textfont=sf]{subfig}
%\else
%  \usepackage[caption=false,font=footnotesize]{subfig}
%\fi
% subfig.sty, written by Steven Douglas Cochran, is the modern replacement
% for subfigure.sty, the latter of which is no longer maintained and is
% incompatible with some LaTeX packages including fixltx2e. However,
% subfig.sty requires and automatically loads Axel Sommerfeldt's caption.sty
% which will override IEEEtran.cls' handling of captions and this will result
% in non-IEEE style figure/table captions. To prevent this problem, be sure
% and invoke subfig.sty's "caption=false" package option (available since
% subfig.sty version 1.3, 2005/06/28) as this is will preserve IEEEtran.cls
% handling of captions.
% Note that the Computer Society format requires a sans serif font rather
% than the serif font used in traditional IEEE formatting and thus the need
% to invoke different subfig.sty package options depending on whether
% compsoc mode has been enabled.
%
% The latest version and documentation of subfig.sty can be obtained at:
% http://www.ctan.org/pkg/subfig




% *** FLOAT PACKAGES ***
%
\usepackage{fixltx2e}
% fixltx2e, the successor to the earlier fix2col.sty, was written by
% Frank Mittelbach and David Carlisle. This package corrects a few problems
% in the LaTeX2e kernel, the most notable of which is that in current
% LaTeX2e releases, the ordering of single and double column floats is not
% guaranteed to be preserved. Thus, an unpatched LaTeX2e can allow a
% single column figure to be placed prior to an earlier double column
% figure.
% Be aware that LaTeX2e kernels dated 2015 and later have fixltx2e.sty's
% corrections already built into the system in which case a warning will
% be issued if an attempt is made to load fixltx2e.sty as it is no longer
% needed.
% The latest version and documentation can be found at:
% http://www.ctan.org/pkg/fixltx2e


%\usepackage{stfloats}
% stfloats.sty was written by Sigitas Tolusis. This package gives LaTeX2e
% the ability to do double column floats at the bottom of the page as well
% as the top. (e.g., "\begin{figure*}[!b]" is not normally possible in
% LaTeX2e). It also provides a command:
%\fnbelowfloat
% to enable the placement of footnotes below bottom floats (the standard
% LaTeX2e kernel puts them above bottom floats). This is an invasive package
% which rewrites many portions of the LaTeX2e float routines. It may not work
% with other packages that modify the LaTeX2e float routines. The latest
% version and documentation can be obtained at:
% http://www.ctan.org/pkg/stfloats
% Do not use the stfloats baselinefloat ability as the IEEE does not allow
% \baselineskip to stretch. Authors submitting work to the IEEE should note
% that the IEEE rarely uses double column equations and that authors should try
% to avoid such use. Do not be tempted to use the cuted.sty or midfloat.sty
% packages (also by Sigitas Tolusis) as the IEEE does not format its papers in
% such ways.
% Do not attempt to use stfloats with fixltx2e as they are incompatible.
% Instead, use Morten Hogholm'a dblfloatfix which combines the features
% of both fixltx2e and stfloats:
%
% \usepackage{dblfloatfix}
% The latest version can be found at:
% http://www.ctan.org/pkg/dblfloatfix




% *** PDF, URL AND HYPERLINK PACKAGES ***
%
%\usepackage{url}
% url.sty was written by Donald Arseneau. It provides better support for
% handling and breaking URLs. url.sty is already installed on most LaTeX
% systems. The latest version and documentation can be obtained at:
% http://www.ctan.org/pkg/url
% Basically, \url{my_url_here}.




% *** Do not adjust lengths that control margins, column widths, etc. ***
% *** Do not use packages that alter fonts (such as pslatex).         ***
% There should be no need to do such things with IEEEtran.cls V1.6 and later.
% (Unless specifically asked to do so by the journal or conference you plan
% to submit to, of course. )


% correct bad hyphenation here
\hyphenation{op-tical net-works semi-conduc-tor}

\newtheorem{theorem}{Theorem}[section]
\newtheorem{lemma}[theorem]{Lemma}
\newtheorem{proposition}[theorem]{Proposition}
\newtheorem{corollary}[theorem]{Corollary}

\newenvironment{proof}[1][Proof]{\begin{trivlist}
\item[\hskip \labelsep {\bfseries #1}]}{\end{trivlist}}
\newenvironment{definition}[1][Definition]{\begin{trivlist}
\item[\hskip \labelsep {\bfseries #1}]}{\end{trivlist}}
\newenvironment{example}[1][Example]{\begin{trivlist}
\item[\hskip \labelsep {\bfseries #1}]}{\end{trivlist}}
\newenvironment{remark}[1][Remark]{\begin{trivlist}
\item[\hskip \labelsep {\bfseries #1}]}{\end{trivlist}}

\newcommand{\qed}{\nobreak \ifvmode \relax \else
      \ifdim\lastskip<1.5em \hskip-\lastskip
      \hskip1.5em plus0em minus0.5em \fi \nobreak
      \vrule height0.75em width0.5em depth0.25em\fi}
      
\usepackage{multirow}

\begin{document}

\newcommand{\mvt}[1]{\vspace*{0.125in}\noindent\textbf{mvt: #1}\xspace}

%
% paper title
% Titles are generally capitalized except for words such as a, an, and, as,
% at, but, by, for, in, nor, of, on, or, the, to and up, which are usually
% not capitalized unless they are the first or last word of the title.
% Linebreaks \\ can be used within to get better formatting as desired.
% Do not put math or special symbols in the title.
\title{Secure Logic Synthesis for Integrated Circuit Obfuscation}


% author names and affiliations
% use a multiple column layout for up to three different
% affiliations


% conference papers do not typically use \thanks and this command
% is locked out in conference mode. If really needed, such as for
% the acknowledgment of grants, issue a \IEEEoverridecommandlockouts
% after \documentclass

% for over three affiliations, or if they all won't fit within the width
% of the page (and note that there is less available width in this regard for
% compsoc conferences compared to traditional conferences), use this
% alternative format:
% 
%\author{\IEEEauthorblockN{Michael Shell\IEEEauthorrefmark{1},
%Homer Simpson\IEEEauthorrefmark{2},
%James Kirk\IEEEauthorrefmark{3}, 
%Montgomery Scott\IEEEauthorrefmark{3} and
%Eldon Tyrell\IEEEauthorrefmark{4}}
%\IEEEauthorblockA{\IEEEauthorrefmark{1}School of Electrical and Computer Engineering\\
%Georgia Institute of Technology,
%Atlanta, Georgia 30332--0250\\ Email: see http://www.michaelshell.org/contact.html}
%\IEEEauthorblockA{\IEEEauthorrefmark{2}Twentieth Century Fox, Springfield, USA\\
%Email: homer@thesimpsons.com}
%\IEEEauthorblockA{\IEEEauthorrefmark{3}Starfleet Academy, San Francisco, California 96678-2391\\
%Telephone: (800) 555--1212, Fax: (888) 555--1212}
%\IEEEauthorblockA{\IEEEauthorrefmark{4}Tyrell Inc., 123 Replicant Street, Los Angeles, California 90210--4321}}




% use for special paper notices
%\IEEEspecialpapernotice{(Invited Paper)}




% make the title area
\maketitle

% As a general rule, do not put math, special symbols or citations
% in the abstract
\begin{abstract}
%Logic locking has emerged in the hardware security literature as a term describing a design-for-trust technique for fighting against hardware piracy. In its most general form, the technique consists of padding the gate-level netlist of a design with elements whose functionality cannot be directly determined by a foundry, unless they are provided with a secret key which is held by the designer. The premise is that without knowledge of the encryption key, a malicious foundry cannot determine the functionality of the original design, thereby making piracy impossible. 

%The technique has been subject to various kinds of attacks, mostly 

%In this paper, we propose the first notion of security for logic encryption, and relate it to concepts from computational learning theory. The proposed notion of security captures the two main objectives of logic encryption, namely those of resiliency to reverse engineering and output corruptibility. The security definition also provides an insight into why some of the published attacks against logic encryption work. We provide an example of a logic encryption scheme that is provably secure according to out definition, and contrast it to existing Our work provides a theoretical framework for the design and analysis of logic encryption techniques, and we hope that it will result in the construction of provably secure algorithms for the problem.

%Traditional attacks on logic encryption schemes exploit the malicious foundry's access to input-output behaviour information of the design, in the form of either a functional IC manufactured in an earlier batch, and purchased from the market, or in the form of test patterns provided by the designer as part of the standard design flow. In this paper, we describe an even stronger attack that , one that does not require access to a functional IC or even test patterns. We 
\end{abstract}

% no keywords




% For peer review papers, you can put extra information on the cover
% page as needed:
% \ifCLASSOPTIONpeerreview
% \begin{center} \bfseries EDICS Category: 3-BBND \end{center}
% \fi
%
% For peerreview papers, this IEEEtran command inserts a page break and
% creates the second title. It will be ignored for other modes.
\IEEEpeerreviewmaketitle
 
\section{Prelimiaries}
A traditional synthesis procedure takes as input a description of some desired circuit behaviour and outputs an implementation of the design in terms of logic gates. We do not treat locking as separate from synthesis; rather we view them as one integrated procedure. We extend the definition of a synthesis procedure by allowing it to receive along with the behavioral description of a circuit; a bitstring representing a key (which we assume the designer have selected uniformly at random from some key space, of size $2^k$, for integer security parameter $k$), and refer to the new procedure as a \emph{locking synthesis procedure}. Rather than outputting a design implementing the circuit behaviour provided as input, a locking synthesis procedure outputs an ``augmented" design, one that satisfies certain correctness criteria. For simplicity, we assume all RTL descriptions are of single-output circuits that consist of only combinational logic, but we note that all definitions and results are easily extend-able to the multiple-output case as well as sequential logic.

\begin{definition}
A \emph{locking synthesis procedure} takes as input a register-transfer level (RTL) description of a circuit, $f$ (which we model simply as text), and a bitstring $k^{*} \in \{0,1\}^{m}$ of length $m$. We denote by $n$ be the number of circuit inputs in the RTL description $f$. For each $(f,k^*)$-pair, the procedure outputs a circuit $c$ with two sets of inputs, $x$ and $k$, of sizes $n$ and $m$ respectively. The circuit $c$  satisfies the following correctness criteria (we use $f: X \rightarrow Y$ and $c: X \times K \rightarrow Y$ where $K = \{0,1\}^{m}$, $X = \{0,1\}^{n}$ and $Y = \{0,1\}$ to denote the Boolean functions of $f$ and $c$ respectively):
$$ c(x,k^{*}) = f(x) \, \, \forall x \in X, $$
and 
$$ \exists  x \in X \,\, s.t. \,\, c(x,k) \neq c(x,k^{*}) \, \, \forall k \in K.$$ 
\end{definition}
%For any $x \in X$, the function outputs $y \in Y$ and we say $y = f(x)$.

%To obfuscate the implementation, the defender introduces $m$ key bits, and selects a key $k^{*} \in \{0,1\}^{m}$ uniformly at random. 

%The function that the defender implements (which the attacker sees) is 
%$c: X \times K \rightarrow Y$ where $K = \{0,1\}^{m}$. 
%For any $x \in X$ and $k \in K$, the function outputs $y \in Y$ and we say $y = c(x,k)$.

%The implementation must satisfy the following:
These requirements guarantee that: (1) when the defender applies the correct key, $k^{*}$,
$c$ provides correct outputs for all inputs; (2) when the attacker applies an incorrect, the resulting circuit differs from the correct one for at least one input.

We extend our definition by allowing a synthesis procedure to be randomized; that is, provided the same circuit description and key as input, two separate executions of the synthesis procedure can result in different outputs (however both must still satisfy the correctness criteria mentioned above).

\section{Secure Synthesis}
Following the principles of secure design, we do not assume a synthesis procedure to be secret; rather we base our security on the secrecy of the correct key $k^*$.

\subsection{A Necessary Condition}

In the following, let $C(f,k)$ be the random variable describing the output of a synthesis procedure when invoked with description $f$ and key $k$ as input.
\begin{definition}
A locking synthesis procedure is secure only if the following condition holds for every behavioural description $f$ and key $k$. For every $k'\neq k$ and $c$ in the support of $C(f,k)$, there exists a behavioural description $f'$ such that
%variables $C(f,k)$ and $C(f',k')$ are statistically indistinguishable. We could opt to require the distributions of $C(f,k)$ and $C(f',k')$ to be perfectly indistinguishable instead; i.e., for each $c$ in the support of $C(f,k)$, 
$P[C(f,k)=c]\leq P[C(f',k')=c]$
%$$ P(K|C)=P(K) $$ 
(probabilities are taken over the synthesis procedure's coin flips). If the synthesis procedure is to satisfy the correctness criteria, such $f'$ cannot be functionally equivalent to $f$.

%A locking synthesis procedure is secure if for a every behavioural description $f$ and key $k$, the synthesis output $C$ and key $K$ satisfy 
%$$ P(K|C)=P(K) $$
%where the probabilities above are computed assuming each of the $2^{2^n}$ behaviours that can be provided as input to the synthesis procedure --- are equally likely.
\end{definition}

This condition does not require $f'$ to be ``likely" to be provided by a chip designer as input to the synthesis procedure, which is why I characterize it as only necessary. It suffices, however, to protect against attacks like ours, where the attacker uses information about the publicly available synthesis procedure to desynthesize the netlist; but does not leverage heuristic knowledge of HDL coding practices.

If we need the procedure to be resilient to attacks where the attacker might leverage domain knowledge of HDL coding practices, we define security as follows.

Let $F_n$ be the set of all Boolean functions $f_n$ of $n$ variables. For Boolean function $f_n$, let $\gamma_{f_n}$ be the set of all behavioural descriptions (of any size) that implement $f_n$. We model a developer of $n$-input circuits as a distribution ensemble $D_n=\{D_{f_n}\}_{f_n\in F_n}$ where distribution $D_{f_n}:\gamma_{f_n}\rightarrow [0,1]$ describes developer $D_n$'s inclination to describe $f_n$ using each element of $\gamma_{f_n}$. For developer $D_n$, let $C_{D_n}$ denote the random variable describing the output of the synthesis procedure (when the function $f_n$ is selected uniformly at random from $F_n$, and a description is sampled according to $D_{f_n}$). A locking synthesis procedure is secure if for every developer $D_n$ it holds that

$$ P(K|C_{D_n})=P(K) $$

where the probabilities above are taken over the coin flips of the procedure and the developer, and $K$ denotes the locking key.
%require the condition above to hold for every behavioural description whose Boolean function is equivalent to that of $f'$ (and of course, that the Boolean functions of $f'$'s are distinct from $f$). Formally, for every $g\equiv f'$, 
%the distributions of $C(g,k')$ and $C(f,k)$ should be perfectly/statistically indistinguishable.
%$P[C(f,k)=c]\leq P[C(g,k')=c]$. 

%We should note however that by doing so, we automatically obtain an indistinguishability obfuscator (if, post synthesis, we set the key bits to their correct values).

\begin{proposition}
If, post synthesis, we set the key bits to their correct values, then a locking synthesis procedure secure according to the definition above functions as an indistinguishability obfuscator.
\end{proposition}

Both notions of security defined above allow a much simpler construction for a secure procedure than our BDD idea. The construction is as follows: XOR $f$'s output with the output of a point function (whose password is the locking key), pass through an indistinguishability obfuscator, XOR the result with the output of a comparator that takes the input and key lines as arguments). The problem here of course is that the output corruptability for an incorrect netlist is very low. The BDD idea could have an advantage here.

%The condition in this definition ensures that given the unencrypted netlist, knowledge of the encrypted netlist does not reduce uncertainty about the locking key --- assuming that every behavioural. 

%The problem it admits every procedure where the 

%This condition is sufficient to protect against an unbounded attacker that cannot make

%If we do not want to make the assumption that that every behaviour is as likely as any other, w

%Is this condition sufficient by itself to guarantee impossibility of key recovery? No. As a counter-example, consider a synthesis procedure that produces, for an input behavioral description, a implementation that, for all keys other than the correct one, behaves as specified by the input description, for all inputs save one (a fixed value for all keys). The construction in Figure?? is one example of such implementation The comparator checks wether the input key . Such synthesis procedure is clearly secure according to Condition 1; the input correct key value is not used at all when constructing the encrypted design; therefore the locked design produced is the same for all input keys, i.e., it is independent of the correct key value. Such a locking 
%We therefore introduce another condition for secure synthesis:
%\begin{proposition}
%A synthesis procedure as defined in Section ?? is secure only if 
%\end{proposition}
%Such a condition rules out procedures like the one in Figure ??.


% conference papers do not normally have an appendix

% trigger a \newpage just before the given reference
% number - used to balance the columns on the last page
% adjust value as needed - may need to be readjusted if
% the document is modified later
%\IEEEtriggeratref{8}
% The "triggered" command can be changed if desired:
%\IEEEtriggercmd{\enlargethispage{-5in}}

% references section

% can use a bibliography generated by BibTeX as a .bbl file
% BibTeX documentation can be easily obtained at:
% http://mirror.ctan.org/biblio/bibtex/contrib/doc/
% The IEEEtran BibTeX style support page is at:
% http://www.michaelshell.org/tex/ieeetran/bibtex/
%\bibliographystyle{IEEEtran}
% argument is your BibTeX string definitions and bibliography database(s)
%\bibliography{IEEEabrv,../bib/paper}
%
% <OR> manually copy in the resultant .bbl file
% set second argument of \begin to the number of references
% (used to reserve space for the reference number labels box)



% that's all folks
\end{document}


